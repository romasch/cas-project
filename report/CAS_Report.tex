\documentclass[11pt, a4paper,oneside,chapterprefix=false]{scrbook}
\DeclareOldFontCommand{\bf}{\normalfont\bfseries}{\mathbf}

\usepackage{a4wide}
\usepackage{times}
\usepackage{helvet}   % sets sans serif font

\usepackage{amsmath,amssymb,amsthm}

\usepackage{graphicx}
\usepackage{subfigure}  
\usepackage{fancybox} % for shadowed or double bordered boxes
\usepackage{fancyhdr}

\DeclareGraphicsExtensions{.pdf, .jpg}

%% macros
\input{include/math}
\input{include/codelisting_layout}

\usepackage{color}
\definecolor{RED}{rgb}{1,0,0}
\definecolor{GREEN}{rgb}{0,0.7,0}
\definecolor{BLUE}{rgb}{0,0,1}
\newcommand{\FIXME}[1]{{\color{RED}{\textbf{FIX}: #1}}}

\addtolength{\textheight}{2.0cm}
\addtolength{\voffset}{-1cm}
\addtolength{\textwidth}{1.8cm}
\addtolength{\hoffset}{-.9cm}

\widowpenalty=10000
\clubpenalty=10000

%\author{Hans Muster}
%\title{Blockwise Hierarchical Data Decompositions}
%\date{Fall Semester 2011}

\begin{document}

\frontmatter
%\maketitle %automatic version
% --- selfmade version ----
\begin{titlepage}
	\setlength{\parindent}{0cm}
	\addtolength{\textheight}{1.0cm}
	\vspace{0.5cm}
	\sffamily\Huge
	{\textbf {CAS Report \\ A comparison of neo4j and postgres}}

	\vfill \vfill \vfill

	\vfill
	\textsf\Large
	CAS Data Science and Machine Learning \\[0.5cm]\large
	21.06.2022\\[0.5cm]
	\large
	by Roman Schmocker, Ergon Informatik AG

	\vfill \vfill \vfill
	\begin{minipage}[b]{0.5\textwidth}
	Supervisors: \\
	Prof. Dr. Sven Helmer
	\end{minipage}
	%
	\begin{minipage}[b]{0.5\textwidth} \raggedleft
	Department of Informatics \\
	University of Zurich
	\end{minipage}

	\vfill
	\hrule
	\vspace{0.5cm}
	\includegraphics*[width=0.3\textwidth]{figures/uzh_logo} \hfill
%	\includegraphics*[width=0.3\textwidth]{figures/vmml_logo}
\end{titlepage}
%%


%=====================================================================
\chapter{Abstract} \label{chp:abstract}
%=====================================================================


{\em Keywords}: TODO

\tableofcontents

\mainmatter

%=====================================================================
\chapter{Introduction} \label{chp:introduction}
%=====================================================================

Over many years the default for data storage was a relational database, and it still is today. 
However, in recent years the NoSQL Movement has gained traction.
NoSQL is actually an umbrella term for many different technologies.
Among them are graph databases.
Graph database optimize graph traversal and path finding and other stuff.
In this report, we want to look at a specific graph database Neo4j and compare it to Postgres, a popular free relational DB.

[TODO maybe also tell why neo4j might be used in our company]

%=====================================================================
\chapter{Database comparison (neo4j vs postgres)} \label{chp:theory}
%=====================================================================

On the relational side we have Postgres, a very classical relational database.
The underlying theory is relational algebra [link to paper].
Data is stored in tables with a strict schema.
The data can be queried using SQL.

Neo4j is a free graph database with commercial support.
Internally all data is stored as a labelled property graph.
Meaning there are nodes with properties, as well as edges with properties.
Internally, neo4j uses the index-free adjacency technique, a storage technique which allows to traverse a graph without looking up nodes in an index.
This is in contrast to relational database where foreign key relations have to be resolved by loading the specific entry from another table through an index.
The data in neo4j is queried through Cypher query language, an SQL like language specifically desigend for graphs.
It allows to match agains graph relations with the --> operator.

%=====================================================================
\chapter{Methods} \label{chp:methods}
%=====================================================================


%---------------------------------------------------------------------
\section{The ICIJ offshoreleaks data set} \label{sec:dataset}
%---------------------------------------------------------------------

The data used for the comparison is from the ICIJ offshoreleaks [link].
The consortium provides the database as a dump, and thus it can be directly imported to neo4j.

For postgres we use the export as CSV function.
This returns the data as one big table, which we can import using the postgres `COPY` statement to a temporary table.
Since nodes and edges are mixed using this technique, we further copy the data along with the relevant properties into a `node` and an `edge` table.
There are no further tables based on type because we want the flexibility of traversing the graph regardless of labels.
(On production systems we may want to have different tables per label or relationship type).

%---------------------------------------------------------------------
\section{Queries} \label{sec:benchmark}
%---------------------------------------------------------------------

There are three queries for the comparison. Note that queries focus a lot on graph traversal and give an advantage to neo4j. 
This is kind of fine, because we consider a scenario where a graph database is used in addition to a relational database.




\subsection{Same Address and Firm}

We want to find two different officers for the same firm who live at the same address, e.g. to find family relations.
This targets the subgraph matching [link?] aspect of graph databases.

\includegraphics*[width=0.5\textwidth]{figures/query_relation.png} \hfill

\subsection{Shortest Path}

Find the shortest path between two nodes, if they are connected at all. E.g. to find out if Marc Rich (former founder of Glencore and always a bit in trouble with the law) still has connections to the Glencore group.

\includegraphics*[width=0.5\textwidth]{figures/query_shortest_path.png} \hfill

\subsection{Transitive Query}

Transitively find all nodes reachable from a start node, maybe up to some level X. E.g. find all entities that are directed by the king of Saudi Arabia [TODO: maybe find a better example]

\includegraphics*[width=0.5\textwidth]{figures/query_transitive.png} \hfill

%=====================================================================
\chapter{Results} \label{chp:results}
%=====================================================================



%---------------------------------------------------------------------
\section{Performance} \label{sec:performance}
%---------------------------------------------------------------------



%---------------------------------------------------------------------
\section{Ease of use} \label{sec:convenience}
%---------------------------------------------------------------------




%=====================================================================
\chapter{Discussion} \label{chp:discussion}
%=====================================================================

%=====================================================================
\chapter{Conclusion and future work} \label{chp:conclusion}
%=====================================================================

\bibliographystyle{alpha}
\bibliography{references}
\end{document}
